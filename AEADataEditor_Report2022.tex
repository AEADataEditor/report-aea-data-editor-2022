% AEJ-Article.tex for AEA last revised 22 June 2011
\documentclass[PP]{AEA}

%%%%%% NOTE FROM OVERLEAF: The mathtime package is no longer publicly available nor distributed. We recommend using a different font package e.g. mathptmx if you'd like to use a Times font.
\usepackage{mathptmx}

% The mathtime package uses a Times font instead of Computer Modern.
% Uncomment the line below if you wish to use the mathtime package:
%\usepackage[cmbold]{mathtime}https://www.overleaf.com/project/5de54d6638f785000167866a
% Note that miktex, by default, configures the mathtime package to use commercial fonts
% which you may not have. If you would like to use mathtime but you are seeing error
% messages about missing fonts (mtex.pfb, mtsy.pfb, or rmtmi.pfb) then please see
% the technical support document at http://www.aeaweb.org/templates/technical_support.pdf
% for instructions on fixing this problem.

% Note: you may use either harvard or natbib (but not both) to provide a wider
% variety of citation commands than latex supports natively. See below.

% Uncomment the next line to use the natbib package with bibtex 
\usepackage{url}
\urlstyle{same} % makes the font the same
\usepackage{natbib}
\usepackage{hyperref}
\usepackage{acronym}
\usepackage[names]{xcolor}
\usepackage{graphicx}
\usepackage{csvsimple}
% Uncomment the next line to use the harvard package with bibtex
%\usepackage[abbr]{harvard}
\usepackage{etoolbox}
\usepackage{geometry}
\usepackage{caption} % to re-use counters
\usepackage{threeparttable}
\usepackage{capt-of}

\newtoggle{fancy}
\togglefalse{fancy}

\newtoggle{draft}
%\toggletrue{draft}
\togglefalse{draft}

\newtoggle{final}
%\togglefalse{final}
\toggletrue{final}
%\iftoggle{final}{\togglefalse{draft}}{}


\iftoggle{draft}{
\usepackage{draftwatermark}
\SetWatermarkText{DRAFT}
\SetWatermarkScale{0.5}
\SetWatermarkLightness{0.85}%
}{\usepackage[final]{draftwatermark}}

\usepackage{xspace}
% to adjust floats
\usepackage{placeins}
% to read the table
\usepackage{booktabs}
\usepackage{ifthen}
\usepackage{csvsimple}
\usepackage{longtable}

\usepackage{textcomp}


\iftoggle{final}{
\usepackage[disable]{todonotes}
\newcommand{\misscitep}[2]{\citep{#2}}
\newcommand{\misscitet}[2]{\citet{#2}}
}{
\usepackage{todonotes}
\geometry{verbose,letterpaper,
	tmargin=1in,bmargin=1in,lmargin=1in,rmargin=2in}
\setlength{\marginparwidth}{1.9in}
\newcommand{\misscitep}[2]{\todo[color=green]{Missing citation: #1}{(\textcolor{red}{#2})}}
\newcommand{\misscitet}[2]{\todo[color=green]{Missing citation: #1}{\textcolor{red}{#2}}}
}



% This command determines the leading (vertical space between lines) in draft mode
% with 1.5 corresponding to "double" spacing.
\draftSpacing{1.5}

%% make somewhat tigher enumeration environments
\usepackage{enumitem}
\setlist[enumerate]{itemsep=0pt,parsep=0pt,topsep=1pt}
\setlist[itemize]{itemsep=0pt,parsep=0pt}

%% Acronyms
\input{acronyms.tex}

% reset colors
\definecolor{darkblue}{rgb}{0 0 255}
\hypersetup{colorlinks,
breaklinks,
citecolor=darkblue,
linkcolor=darkblue,
urlcolor=darkblue}

% Different ways to cite URLS
%\newcommand{\urlcite}[2]{\href{#1}{#2}
\newcommand{\urlcite}[2]{#2\footnote{\url{#1}}}
\newcommand{\purlcite}[2]{#2.\footnote{\url{#1}}}
\newcommand{\curlcite}[2]{#2,\footnote{\url{#1}}}
\newcommand{\furlcite}[2]{#2 (\url{#1})}

% redefine subparagraph

\renewcommand{\subparagraph}[1]{\textbf{#1}}

%
% Periods covered by the report 
% Should be read in from R
\newcommand{\version}{Thu Mar  2 03:41:48 2023}
\newcommand{\teamsize}{41}
\newcommand{\firstday}{2021-12-01}
\newcommand{\lastday}{2022-11-30}
\newcommand{\mcpubnoncompl}{12}
\newcommand{\mcpubupdates}{13}
\newcommand{\jiraissues}{500}
\newcommand{\jiramcs}{429}
\newcommand{\jiraissuescplt}{361}
\newcommand{\jiraexternal}{10}
\newcommand{\jiramcscplt}{316}
\newcommand{\jiramcspending}{257}
\newcommand{\jiramcsexternal}{10}
\newcommand{\medianrounds}{1}
\newcommand{\pmedianrounds}{1}
\newcommand{\roundone}{91.9}
\newcommand{\proundone}{77.8}
\newcommand{\roundthree}{NA}
\newcommand{\proundthree}{0.7}
\newcommand{\pkgcount}{427}
\newcommand{\pkglastday}{2022-11-15}
\newcommand{\pkgsizetwog}{15}
\newcommand{\pkgsizetwentyg}{3}
\newcommand{\pkgsizetotalgb}{951.48}
\newcommand{\pkgsizetotaltb}{1}
\newcommand{\pkgsizemean}{2281.77}
\newcommand{\pkgsizemedian}{30.15}
\newcommand{\pkgsizeqsvntyfv}{635.2}
\newcommand{\pkgfilesT}{51}
\newcommand{\pkgfiles}{51394}
\newcommand{\icpsrutilization}{2022-12-07}
\newcommand{\icpsrtotalDownloads}{297660}
\newcommand{\icpsrtotalDownloadsHT}{3}
\newcommand{\icpsrmaxDownloads}{1889}
\newcommand{\icpsrmedianDownloads}{37}
\newcommand{\icpsrtotalViews}{2527869}
\newcommand{\icpsrtotalViewsM}{2.5}
\newcommand{\icpsrtotalPublished}{4171}
\newcommand{\zenodototalDownloads}{284}
\newcommand{\zenodototalDownloadsHT}{0}
\newcommand{\zenodomaxDownloads}{119}
\newcommand{\zenodomedianDownloads}{11}
\newcommand{\zenodototalViews}{633}
\newcommand{\zenodototalViewsM}{0}
\newcommand{\zenodototalPublished}{9}
\newcommand{\zenodototalVolumeGB}{1501}
\newcommand{\zenodototalVolumeTB}{1.5}
\newcommand{\zenodototalSizeGB}{373}
\newcommand{\zenodototalSizeTB}{0.4}
\newcommand{\zenodototalFiles}{6163}

\newcommand{\versionreg}{Sun Dec 11 20:55:05 2022}
\newcommand{\registeredusers}{8300}
\newcommand{\activeusers}{3660}
\newcommand{\regsyearly}{1308}
\newcommand{\regscumul}{6700}


%\renewcommand{\firstday}{Dec 1, 2019}
%\renewcommand{\lastday}{Nov 30, 2020}

\begin{document}

\title{Report for 2022 by the AEA Data Editor }
\shortTitle{Report by Data Editor}
\author{Lars Vilhuber\thanks{%
Cornell University, lars.vilhuber@cornell.edu. }
}
\date{\today}
\pubMonth{Dec}
\pubYear{2022}
\pubVolume{--}
\pubIssue{--}
\JEL{}
\Keywords{reproducibility; replicability; science of science}




\maketitle

The \ac{AEA} Data Editor's  mission is to ``design  and  oversee  the  AEA  journals’  strategy for archiving and curating research data and promoting  reproducible  research'' \citep{10.1257/pandp.108.745}. The 2018 Report by the Data Editor \citep{10.1257/pandp.109.718} articulates how to implement that mission. 
Since July 2019, we have conducted comprehensive pre-publication reproducibility checks for all regular AEA journals, developed guidance for authors, and worked with peers at societies and groups in economics and elsewhere. We currently conduct  pre-publication reproducibility checks for all AEA journals, and conduct  basic checks on replication packages for Papers and Proceedings. General policy and various auxiliary policies are listed in Appendix~\ref{sec:list-of-policies}.

We also reach out to numerous data creators and providers --- both authors who have created unique data resources, and academic and commercial data providers that often provide the data for economic research --- and  discuss access to data for reproducibility checks, mechanisms to request publication approval with them, as well as generally inform them of the need for reproducibility, provenance tracing, and transparency in economic research. In some cases, we provide guidance on how to make data publication compliant with FAIR practices \citep{FORCE11FAIRDATAPRINCIPLES}, and assist them in finding additional resources. 
We continue to coordinate with other journals, societies, and registries on all of these topics.

This report also discusses progress and developments at the \rctr{}  (see Section~\ref{sec:registry}),  on behalf of the AEA Oversight Committee for Registry of Random Controlled Trials. 


\section{Infrastructure for Verification of Reproducibility}
\label{sec:infrastructure}

The Data Editor manages the infrastructure needed to access data and code, conducts reproducibility checks, and archives and preserves replication packages. In general, the first two infrastructure pieces are provided by the replication team at Cornell University, the latter primarily by the  \aeadcr{} provided by openICPSR at the University of Michigan, with additional support from the AEA's in-house IT staff. In 2022, the Data Editor continued to explore the use of several other infrastructures for  conducting reproducibility checks and for the preservation of data for replication packages.


\subsection{Pre-publication verification of computational reproducibility}
\label{sec:verification}

\paragraph{The process}

Pre-publicaton verification is conducted by the Data Editor's team at Cornell University. 
Requests for assessment of reproducibility are received and assigned to a team member, who then assesses data availability and compliance with requirements. When some data are available, a full or limited reproducibility check is conducted. If we cannot obtain access to the data or computational resources in a timely fashion, we may reach out to third-parties who can, and request a reproducibility check from them. Once all computations have been completed, a process that can take anywhere from a few minutes to several weeks, a report is compiled, reviewed and approved by the Data Editor, and submitted back to journal editors, who handle most communications with the authors. The report will have  one of four possible recommendations (see Table~\ref{tab:responses}). A ``conditional acceptance'' implies that a revision will need to be resubmitted to the Data Editor to address any identified shortcomings. An ``acceptance'' means that no further changes are necessary, and both the manuscript (after copy-editing) and the replication package can be scheduled for publication.%
%
\footnote{Manuscript and replication package are generally published at the same time, though at the request of either editors or authors, the replication package can be published at any time after acceptance.} 
%
However, to streamline processing, we may also recommend an ``acceptance with modifications requested.'' In such cases, the remaining modifications are minor, and can be handled during copy-editing (for instance, a small number of tables need minimal changes) and prior to publication of the replication package (for instance, a fixable error in a program, or a clarification in the README, not affecting any important tables or figures). 


%\begin{table}[t]
\begin{center}
	\captionof{table}{Recommendations}{}
	\label{tab:responses}
	\centering
	
% Table created by stargazer v.5.2 by Marek Hlavac, Harvard University. E-mail: hlavac at fas.harvard.edu
% Date and time: Thu, Mar 02, 2023 - 03:41:54 AM
\begin{tabular}{@{\extracolsep{5pt}} lr} 
\toprule 
Response option & Frequency \\ 
\midrule Accept & 48 \\ 
Accept - with Changes & 231 \\ 
Conditional Accept & 32 \\ 
Revise and Resubmit & 5 \\ 
\bottomrule 
\end{tabular} 

 \end{center}
%\end{table}

We have increasingly made use of this feature. While we check that authors comply with the request for modifications, no further computational assessment is made. A recommendation of  ``revise and resubmit'' is recorded when we receive a request prior to a conditional acceptance, i.e., during the ``R\&R'' phase. When we have serious concerns, we will reach out directly to the responsible editor, and discuss solutions with the authors. 

\paragraph{Assessments made}

Between \firstday{} and \lastday{}, the AEA Data Editor team  received
\jiraissues{} requests,  for \jiramcs{} manuscripts.%
%
\footnote{This includes only requests submitted between those dates, and does not take into account in-progress requests on \firstday{}.}
%
Requests typically are channeled to the team by the AEA's journal submission and review system, but others were initiated by authors or editors directly, often while preparing the replication materials. Of these,  \jiraissuescplt{} reports (\jiramcscplt{} manuscripts) were submitted back to editors,\footnote{The balance are either in progress or are not coded in the adminstrative system as having been submitted to ScholarOne, such as replication packages for Papers and Proceedings.} and \jiramcspending{} were completed up to the point of publication of the data deposit, including any post-acceptance modifications.  Table~\ref{tab:responses} shows the distribution of the last recommendation on record for manuscripts as of \lastday{}.  Table~\ref{tab:jirastats} breaks these numbers down by journal, showing the number of requests received (``rcvd'') and  reports completed (``cplt'') in the left panel. The right panel shows the number of manuscripts for which one or more requests were received (``rcvd'') and reports completed (``cplt''). The columns marked ``ext.'' identify cases where we reached out to external replicators, which we discuss later. Finally, the last column identifies manuscripts for which the entire process has been completed, and which are ``pending'' publication.
%

\begin{table}[]
    \caption{Processing Statistics}
    \label{tab:jirastats}
    \small
    \begin{threeparttable}
    \centering
    
% Table created by stargazer v.5.2 by Marek Hlavac, Harvard University. E-mail: hlavac at fas.harvard.edu
% Date and time: Thu, Dec 08, 2022 - 04:38:18 PM
\begin{tabular}{@{\extracolsep{5pt}} lrrrrrrr} 
\toprule 
        & \multicolumn{3}{c}{Issues} & \multicolumn{4}{c}{Manuscripts}\\
        \cmidrule{2-4}\cmidrule{5-8}
Journal &  (rcvd) &  (cplt) &  (ext.) &  (rcvd) &  (cplt) &  (ext.) &  (pend.) \\ 
\midrule 
AEJ:Applied Economics & 112 & 91 & --- & 72 & 67 & --- & 29 \\ 
AEJ:Economic Policy & 57 & 48 & 2 & 54 & 45 & 2 & 25 \\ 
AEJ:Macro & 52 & 45 & 4 & 45 & 41 & 4 & 21 \\ 
AEJ:Micro & 20 & 18 & 1 & 18 & 16 & 1 & 6 \\ 
AER & 103 & 83 & 1 & 96 & 78 & 1 & 64 \\ 
AER:Insights & 33 & 28 & 1 & 30 & 26 & 1 & 19 \\ 
JEL & 26 & 22 & --- & 21 & 19 & --- & 15 \\ 
JEP & 27 & 26 & 1 & 24 & 24 & 1 & 21 \\ 
Totals & 500 & 361 & 10 & 429 & 316 & 10 & 257 \\ 
\bottomrule 
\end{tabular} 

    \begin{tablenotes}
    \item[] \textit{Notes:} Data for requests received by the AEA Data Editor between \firstday{} and \lastday{}. AEA P\&P are excluded from this table. See text for details.
    \end{tablenotes}
 \end{threeparttable}
   \end{table}





\paragraph{Issues encountered}


\subparagraph{Incomplete data provenance and data availability:} Most articles still provide imprecise or incorrect information regarding access to data that is not provided. In some cases, authors fail to provide data that should be provided, and in other cases, authors inadvertently provide data for which they do not have redistribution rights. Our impression is that there is improvement over the year, but we do not formally capture this as part of our reporting.

\subparagraph{Specification of computational environment:}
Sufficiently precise descriptions of the required auxiliary packages or libraries, as ``manifest''-like files in R, Python, and Julia, or as "\texttt{setup.do}"-like programs in Stata remain rare. This has not changed much either, and there remains much room for improvement. We have seen only a handful of packages use containers (``Docker''), and we have used containers in a few additional verifications. We continue to work with some users to leverage such environments (see our discussion later under \textit{Computational Infrastructure}).

\subparagraph{Incomplete instructions and manual manipulation:} Heuristically, the number of manual instructions to run code, or to save tables and figures, which detract from speedy and efficient reproduction by third parties, remain too high. 


\paragraph{Delays} 

A recurring concern expressed by authors, editors, and staff members are delays in publication, due to the verification process. The median manuscript is reviewed once (Table~\ref{tab:pre:rounds} shows the breakout by journal).
%
We have continued to reduce the number of rounds before a paper is accepted by accepting replication packages subject to minor post-acceptance edits (the ``accept with changes'' decision described earlier). Figure~\ref{fig:rounds} illustrates the difference graphically, by journal. This comes with additional challenges, since authors are less responsive for subsequent (but still necessary) change requests prior to publication. 

Additional delays were encountered this year due to training and hiring challenges. We train three times a year for acceptance in the LDI Lab, and continue to have an excellent retention rate of lab members once trained. However, at each of this year's training events, students have been unable to attend at the last minute. Failure to attend the intensive one-day training is highly correlated with later retention and efficiency. We are addressing these issues through increased recruiting for the training, and expansion of the pool of participants.



\begin{table}
    \centering
    \caption{Assessment rounds for completed manuscripts}
    \label{tab:pre:rounds}
    \begin{threeparttable}
    \centering
    
% Table created by stargazer v.5.2 by Marek Hlavac, Harvard University. E-mail: hlavac at fas.harvard.edu
% Date and time: Thu, Mar 02, 2023 - 03:41:56 AM
\begin{tabular}{@{\extracolsep{5pt}} ccccccc} 
\toprule 
Rounds & AER & AER Insights & AEJ Applied & AEJ Macro & AEJ Micro & AEJ Policy \\ 
\midrule 1 & 64 & 20 & 37 & 32 & 12 & 38 \\ 
2 & 7 & 2 & 3 & 3 & 0 & 3 \\ 
\bottomrule 
\end{tabular} 

    \begin{tablenotes}
    \footnotesize
    \item[] \textit{Notes:} Data for papers first sent to the AEA Data Editor between \firstday{} and \lastday{}, and for which all rounds have been completed. AEA P\&P, JEP, and JEL are excluded from this table. See text for details. Numbers differ slightly between this table and Table~\ref{tab:jirastats} because they are extracted from two different administrative systems, with different timing cutoffs.
    \end{tablenotes}
 \end{threeparttable}
\end{table}

\begin{figure}
    \includegraphics[width=\textwidth]{images/plot_rounds_compare.png}
    \centering
    \caption{Comparing rounds per journal between 2021 and 2022\label{fig:rounds}}
\end{figure}

\paragraph{Disseminating process information}

We continue to improve our documentation, based on careful monitoring of the process, and where more information could be beneficial. Our documentation aims to (a) provide authors with the information as early as possible, when it is still easy to include reproducible practices in projects at relatively low cost and (b) provide authors with the best information, to reduce frictions and uncertainty. Authors are provided with an informational form upon submission, and a short form, provided upon conditional acceptance, collects salient information about the replication package, but also links to important guidance.%
\footnote{These forms can also be found at \href{https://www.aeaweb.org/journals/data}{aeaweb.org/journals/data}.} 
We now require that authors provide the information as per the Social Science Data Editors' template ``README'' \citep{READMEv1.1.0}, an improved version of which  was introduced in November 2022 after feedback from authors and consultation amongst multiple data editors (see Section~\ref{sec:infrastructure}). 

\paragraph{Computational Infrastructure}

Most replication packages are computationally verified by replicators on the computers available to the Data Editor at the Cornell University Economics Department and the ILR School. The majority are handled on the Windows Server systems of the Cornell Center for Social Sciences, while some are run on the Linux-based Bioinformatics cluster. Occasionally, personal macOS laptops are used. Systems can handle memory requirements up to 1024 GB or up to 100 cores. 

While these systems are fairly standard, they cannot cover all scenarios described in authors' computational requirements. Furthermore, these systems, much like the authors' own systems, are not shareable more broadly, and thus sometimes make it difficult to control for specific requirements, or to share error messages in the most reproducible way.  

We continue to leverage additional computational environments. We have used \curlcite{https://codeocean.com}{CodeOcean} \citep{clyburne-sherin_computational_2019} both to share active (but only partially successful) reproduction efforts and to publish reproducible ``capsules.'' We collaborate with the team behind ``WholeTale'' \citep{BrinckmanFutureGener.Comput.Syst.2018} and the Odum Institute, which conducts reproducibility checks for various political science journals. Both {{CodeOcean}} and WholeTale rely on containerization, often known under the commercial name ``Docker,'' which can be independently used to precisely define and then share computational environments. We use containers through CodeOcean and or WholeTale, when appropriate. We use Docker on the Linux cluster at Cornell University, or on Linux workstations or macOS laptops amongst the LDI lab members when storage or compute resources are insufficient at the public providers. Sample code  can be found at the \purlcite{https://github.com/AEADataEditor/}{AEA Data Editor's Github repository}. Pre-configured Stata and manuscript-specific Docker images can be found at Docker Hub.\footnote{Generic images are at \href{https://hub.docker.com/u/dataeditors}{hub.docker.com/u/dataeditors}, specific images at \href{https://hub.docker.com/u/aeadataeditor}{hub.docker.com/u/aeadataeditor}.} A more expansive overview of containerization issues in economics can be found at \citet{aea_data_editor_use_2021}.


\subsection{Archive for Replication Packages}


\begin{table}[t]
    \centering
    \caption{Deposit statistics}
    \label{tab:webstats}
     \begin{threeparttable}
     
% Table created by stargazer v.5.2 by Marek Hlavac, Harvard University. E-mail: hlavac at fas.harvard.edu
% Date and time: Thu, Mar 02, 2023 - 03:42:01 AM
\begin{tabular}{@{\extracolsep{5pt}} ccccccc} 
\toprule 
Repository & Published & Downloads & Views & Uploads & Files & Size (GB) \\ 
\midrule ICPSR & 4,171 & 297,660 & 2,527,869 & 427 & 51,394 & 951.48 \\ 
Zenodo & 9 & 284 & 633 & 4 & 6,163 & 372.6 \\ 
\bottomrule 
\end{tabular} 

 
    \begin{tablenotes}
    \footnotesize
    \item[] \textit{Note}: Unit of observation are deposits at the named repository. Columns 1-3 are for all currently published deposits as of \pkglastday{}. Columns 4-6 are for deposits made between \firstday{} and \pkglastday{}. The number of uploads may not correspond to the number of manuscripts processed by the Data Editor team. Not all uploads have been published yet. 
    \end{tablenotes}
    \end{threeparttable}
\end{table}


The default archive for replication packages accompanying articles in AEA journals is the \aeadcr{}. Deposit instructions are provided on the Data Editor's website, and mentioned upon conditional acceptance. However, it is not the only acceptable archive, as we discuss below.


Table~\ref{tab:webstats} shows statistics for all currently published replication packages at the \aeadcr{}.  There are currently  
\icpsrtotalPublished{} published replication packages.  Between \firstday{} and \pkglastday{}, \pkgcount{} deposits were made, with over \pkgfilesT{} thousand files and nearly 1 TB of data. 

In the past year, the median package size was  \pkgsizemedian{} MB, but a significant number of packages (\pkgsizetwog{} percent) had  packages larger than 2GB. \pkgsizetwentyg{} percent of deposits were larger than 20GB. Note that any partial data packages over 30 GB, the default quota for a deposit, are archived elsewhere. One possible repository is Zenodo, where the  ``AEA Zenodo community'' is available. Table~\ref{tab:webstats} shows a small number of very large packages, with \zenodototalSizeGB{}~GB of data already deposited in only \zenodototalPublished{} packages. 

Some packages have more than 1,000 files, hitting a technical constraint on openICPSR. Provision of opaque ZIP files are generally prohibited. Instructions on how to proceed when file numbers are large, while maintaining maximum visibility onto the file and package structure, are provided on the website. Authors with large packages, or packages with more than 1,000 files, should contact the AEA Data Editor. Depositing at other trusted repositories is one option, described in the next section.

Since the migration to the \aeadcr{}, these replication packages have been viewed \icpsrtotalViewsM{} million times and downloaded nearly \icpsrtotalDownloadsHT{} hundred thousand times.




%\subsection{Migrating Historical Supplements}
%\label{sec:migration}
%
%We  migrated the bulk of historical data (and code) supplements in 2019, from ZIP files stored on AEA servers to the \aeadcr{}. A few dozen replication packages are still awaiting migration. Most of these packages either have a very large number of files that surpass technical limitations of the \aeadcr{}, or turned out to be corrupted or illegible in their original version. We are still working on migrating these as resources permit. 
%
%% report on UMich metadata project
%% data provided by email
%
%We reported last year on a project with a research team at the University of Michigan, which solicited metadata improvements to migrated replication packages.  911 researchers provided additional metadata for 522 studies, which were incorporated into the \aeadcr{} in October 2021. We are currently assessing the impact of the improved metadata on findability of such packages.



\begin{figure}[t]
    \centering
    \includegraphics[width=\textwidth]{images/plot_filesize_dist.png} 
    \caption{Size distribution of replication packages deposited at openICPSR between  \firstday{} to \lastday{}, top-coded at 10GB.}
    \label{fig:size_packages}
\end{figure}


\subsection{Third-party repositories}

The \ac{DCAP} allows for code and data  to be deposited at other trusted repositories, as long as all other elements of the \ac{DCAP} are complied with. In fact, authors are \textit{discouraged} from duplicating deposits they have made elsewhere. This is intended to allow authors to create replication packages prior to submitting at the AEA's journals, or any other journal, as a component of a reproducible workflow and possibly in compliance with funder data management policies. Examples of other repositories include the \urlcite{https://dataverse.harvard.edu/}{Harvard Dataverse} and \purlcite{https://zenodo.org/}{Zenodo}. Authors depositing on Zenodo can request inclusion in the ``AEA Zenodo community'' at \href{https://zenodo.org/communities/aeajournals/}{zenodo.org/communities/aeajournals/}. In many of these cases, the Data Editor has actively assisted authors in preparing  data archives, and shared tools that make such data publication easier (see also next section).%
\footnote{Code to support uploading large quantities of data to Zenodo via the Zenodo API, originally created by LDI Lab Member Vansh Gupta, can be found at \href{https://github.com/AEADataEditor/Upload-to-Zenodo}{github.com/AEADataEditor/Upload-to-Zenodo}.}  Third-party repositories are linked to the main \aeadcr{} deposit, and are cited in the main article when appropriate. Authors wishing to deposit replication packages early in the research lifecycle are encouraged to consult the \curlcite{https://social-science-data-editors.github.io/}{Social Science Data Editors website} where links to trusted repositories are provided.  

\input{sec_support}

\section{Working with Other Providers of Scientific Infrastructure to Improve Support for Documenting Provenance and Replicability}
\label{sec:coordination}

An important responsibility of the AEA Data Editor is to interact with other providers of scientific infrastructure. This includes other publishers and journals, archives such as ICPSR, providers of restricted or proprietary data, metadata harvesters, and third-party verification services. 

\subsection{AEA RCT Registry}
\label{sec:registry}

The \rctr{} (Registry) provides services to the economics community at large. Managed at J-PAL and funded by the AEA, registration at the Registry is mandatory for field experiments published in \ac{AER}, \ac{AERI}, \ac{AEJAPP}, and \ac{AEJPOL}, but is also used more broadly in the economics discipline, with numerous publications in top economics journals as well as field journals identifying pre-registration in the Registry. The Registry is available at \href{https://www.socialscienceregistry.org/}{www.socialscienceregistry.org}.

In collaboration with members of the AEA's \textit{Oversight Committee for Registry of Random Controlled Trials}, the AEA Data Editor and the Registry team at J-PAL have worked to improve the usability of the Registry as well as the availability of Registry data. Over the course of 2022, the Registry team rolled out numerous improvements to the infrastructure of the Registry website. While many of these were back-end improvements to keep the web app up-to-date, functional, and secure, a number were front-end changes to improve the experience of registrants and users searching for trials. A few of these are worth noting:
\begin{itemize}
    \item Updates to the version change system to make it easier for users to understand what changed from one version of a registration to the next.
    \item Improved documentation, including an expanded FAQ section, more information in the registration itself, and updated documentation on the registry data snapshots archived on Dataverse \citep[see f.i.][]{DVN/TGMJFD_2022}.
    \item Improvements to the file upload process to make it easier for registrants to upload documents, including (but not limited to) PAPs.
\end{itemize}
Over the course of the summer and fall of 2022, the Registry team interviewed over 60 users, soliciting views on potential improvements to the website. The results are being used to compile  a list of priority projects for next year. Suggested improvements include further improving registration guidance, increasing the registry’s connection with existing citation and index systems, including Crossref, Google Scholar, and RePEc, and providing a pre-fill option for updating registrations with post-trial information, including linked publications. 

Data for the Registry is being curated and preserved in the \textit{AEA RCT Registry Dataverse} at 
\href{https://dataverse.harvard.edu/dataverse/aearegistry}{dataverse.harvard.edu/dataverse/aearegistry}, allowing for reproducible analysis of the universe of registrations by researchers.%
\footnote{Data shown in this report are derived from \citet{DVN/TGMJFD_2022}. End-of-year numbers for 2022 were extrapolated based on data from the first 11 months of 2022, as of 2022-12-01. }

The Data Editor monitors proper reference to and citation of registrations. A registration should be cited in the text and the title footnote as, for example, ``\textit{Zhang (2017)}'' (the title footnote should also mention ``\textit{AEARCTR-0000005}''), and the references should have the appropriate entry:

\begin{quote}\footnotesize
    Zhang, Kelly. 2017. "Voter Pessimism and Electoral Accountability: Experimental Evidence from Kenya." AEA RCT Registry. May 02. https://doi.org/10.1257/rct.5-8.0.
\end{quote}

%{\color{red}Additional data to come.}

While the AEA journals mentioned above remain among the few economics journals that strictly require registration, usage of the registration continues to increase strongly. As Figures~\ref{fig:registry1} and~\ref{fig:registry2} show, the rate of registrations per year continues to rise, and the registry now has over \regscumul{} registrations, which are being added at a rate of \regsyearly{} per year.  More than \registeredusers{} unique researchers are associated with these registrations (left panel, Figure~\ref{fig:registry2}), \activeusers{} of which are associated with registrations that have been active in the past 12 months. The share of pre-registrations, favored by some, surpassed post-registrations for the first time in 2021, and remains higher (right panel, Figure~\ref{fig:registry2}). 

% src=https://github.com/J-PAL/AEA_registryanalysis/archive/refs/heads/main.zip

\begin{figure}
\centering
\includegraphics[width=0.4\textwidth]{data/registry/AEA_registryanalysis-main/Output/reg_pre_year.png}
\includegraphics[width=0.4\textwidth]{data/registry/AEA_registryanalysis-main/Output/reg_cumulative.png}
\caption{Annual (left) and Cumulative Registrations (right) }
    \label{fig:registry1}
\end{figure}

\begin{figure}
\centering
\includegraphics[width=0.4\textwidth]{data/registry/AEA_registryanalysis-main/Output/registered_users.png}
\includegraphics[width=0.4\textwidth]{data/registry/AEA_registryanalysis-main/Output/post_pre_reg.png}
\caption{Unique registered investigators (left), Post vs Pre-registrations (right)}
    \label{fig:registry2}
\end{figure}

\subsection{Highlighting and Preserving Data Resources}

We identified Zenodo as a possible solution to preserving large-scale data resources that surpass the practical capabilities of the \aeadcr{}. By creating the ``AEA Zenodo Community,'' we are able to highlight data resources that have, in the past, often not been curated or preserved appropriately. For instance, \citet{u_s_geological_survey_2022_5830968} comprises 83GB of data, preserved from DVDs and complemented and corrected manually based on online sources. They constitute some of the input data for \citet{10.1257/app.20200398}. Similarly, \citet{ministerio_de_desarrollo_productivo_arge_2022_6568295} preserves 44 GB of daily price data  captured in 2016-2018 and used by \citet{10.1257/mac.20210172}. In both cases, the data are under liberal licenses (public domain or Creative Commons licenses), which allow for the republication of the data. 


\subsection{Data Providers}
\label{sec:producers}

We regularly meet and communicate with academic, governmental, and commercial data providers, on behalf of specific authors or because we have identified a data provider as a frequently used resource. Discussion topics include making data citations easier, clarifying licenses, requesting blanket or streamlined data redistribution or access authorizations, or suggesting improved data curation practices to avoid repeatedly copying data from uncurated websites to the curated \aeadcr{}. 

In the course of the past year, we have talked about some or all of these topics with IPUMS, the World Bank, staff at the U.S. Census Bureau, the \ac{IRS}, and the \ac{BLS}, Philipp vom Berge and Michael Oberfichtner at the (German) Institute for Employment Research (IAB), Paulo Guimarães at the \ac{BPLIM},  and Kamel Gadouche  at the French \textit{Centre d'accès sécurisé de données} (CASD). We have also talked to various research groups on how to improve data curation, visibility, and citability of the data created by their efforts. 

\subsection{Economics Journals}

We continue to coordinate with other data editors conducting similar activities at other journals. An informal mailing list managed by the AEA Data Editor is used to interact with others.\footnote{Journal editors are encouraged to join the mailing list by contacting the AEA Data Editor.} Mailing list members who wish to be more actively involved can participate in the development of the \purlcite{socialsciencedataeditors.github.io}{website of the Social Science Data Editors} In addition to the Data Editor of the AEA, the current group includes Miklos Kóren (Data Editor, \acl{ReStud}), Joan Llull (Data Editor, \acl{EJ} and Econometric Journal), Marie Connolly (Data Editor, \acl{CJE}), Anna Dreber Almenberg (Editor, JPE Microeconomics), and Andrea Weber (Editor, \acl{JEEA}). The website contains guidance on data citations and data availability statements, best practices for coding and data preparation, and links to various tools useful to replicators. In particular, the group coordinates the \textit{Template README} \citep{READMEv1.1.0}, which was last updated in November 2022, and which is a \textit{common} requirement across a number of journals. Furthermore, the group has developed a common standard for replication packages, which journals can use to signal that their requirements are similar to those commonly required by all endorsers of the standard \citep{datacodestandardv1}. The Data Editor  also  participates in  the Steering Committee of a group of data repository leaders at Data-PASS organized as the Journal Editors Discussion Interface (\urlcite{https://dpjedi.org/}{JEDI}).


\subsection{Third-party verification services}
\label{sec:3rdparty}


We continue to rely on and have discussions with third-party verification services. As noted earlier, \jiraexternal{} reports were provided by external replicators or replication services
%for \jiramcsexternal{} manuscripts 
(see Table~\ref{tab:jirastats} for statistics by journal, and Appendix~\ref{app:3rdparty} for a list of third-party replicators). 

\section{Working with the Economics Community to Enhance and Broaden Education on Replicable Science}

Outreach through presentations and publicly available tools is a key component of an effective data and code availability policy. 
%
Recordings of  presentations (when available) and presentation materials are listed at the \purlcite{https://aeadataeditor.github.io/talks/}{Data Editor's website}. In addition to presentations, the Data Editor occasionally participates in \textit{Replication Challenges} or \textit{Replication Games}, where teams of researchers and students attempt to reproduce and extend papers published in leading economics journals. Such post-publication verifications are a necessary and useful check on published replication packages. Replicators may find issues that were not discovered, or discoverable, by pre-publication replicators such as the AEA Data Editor team. The Data Editor facilitates and monitors that any corrections or suggested improvements are conveyed to authors, and are reflected on replication packages via the AEA's \textit{Policy on Revisions of Data and Code Deposits} (see Appendix~\ref{sec:list-of-policies}). 
%
The training of replicators for the AEA's Data Editor team was published as \citet{vilhuber_teaching_2022}.

\subsection{Resources}

The AEA Data Editor maintains public resources available to the economics community. These are made available through a dedicated website at \href{https://aeadataeditor.github.io/}{aeadataeditor.github.io/} and code and project templates provided at \href{https://github.com/aeadataeditor}{github.com/aeadataeditor}. In particular:

\begin{itemize}
    \item Step-by-step guidance on how to prepare a replication package is provided at \href{https://aeadataeditor.github.io/aea-de-guidance/}{aeadataeditor.github.io/aea-de-guidance/}, including video tutorials and a description of the process. 
    \item The template README \citep{READMEv1.1.0} is referenced as part of the guidance, and separately accessible at \href{https://social-science-data-editors.github.io/template_README/}{social-science-data-editors.github.io/template\_README/}.
    \item Various blog posts on topics relating to computational reproducibility are posted at \href{https://aeadataeditor.github.io/year-archive/}{aeadataeditor.github.io/year-archive/} and typically summarized on Twitter under the Data Editor's handle \href{https://twitter.com/AEAData}{@AEAData}, and more recently on Mastodon under \href{https://mstdn.social/@aeadata}{@aeadata@mstdn.social}
    \item Instructions to replicators for assessing authors' replication packages are provided at \href{https://github.com/AEADataEditor/replication-template}{github.com/AEADataEditor/replication-template}.
    \item Template code for using containers for Stata, R, Julia, and Gurobi can be found by \href{https://github.com/AEADataEditor?q=docker&type=all&language=&sort=}{searching for ``docker`` on the Github site}.
    
\end{itemize}


\section{Replication team at Cornell University}

\subsection{Replicators} 
\label{app:replicators}

The following \textbf{\teamsize} students have provided excellent assistance in reproducing the results from the \jiramcs{} articles processed by the Replication Lab:
%
% Pulled from processing-jira...
%
%
Akshay Yadava,
Alizay Zartash,
Ananya Bakshi,
Andres Aradillas Fernandez,
Anjini Khanna,
Ashley Yu,
Bianca Jimenez,
Botao Yao,
Cade Lenczycki,
Christine Cho,
Daniella Pena,
Dhilan Bansal,
Edward Vu,
Eli Kolodezh,
Emma Sbrollini,
Ilona Khimey,
Jacob Brogdon,
Jacob Recht,
Jaeyoung Shim,
Jared Martin,
Jason Lan,
Jessica Rizzo,
Jonathan Temkin,
Julia Zimmerman,
Kate Hofer,
Kevin Bao,
Kushal Kumar Reddy,
Leslie Geng,
Lilly Thomalla,
Lincy Chen,
Manvir Chahal,
Matthew LaFontaine,
Melanie Brown,
Miranda Zhou,
Nathan Maidi,
Raymond Wang,
Satya Datla,
Suvd Khaliun,
Weiting Shen,
Yicheng Yang.



Graduate students  Leonel Borja Plaza and Linda Wang and Research Aides Michael Darisse and Sofia Encarnacion (all Cornell University) have been invaluable assistants in training and coordinating the work as well as developing the methods and procedures which we have made public. Michael has left to pursue a Ph.D. in economics, and Leo has left the lab to concentrate on completing his Ph.D. Linda Wang contributed programming to this report. 

\subsection{Computing support}

We thank the Economics Department and the ILR School for providing us with computing resources at the Cornell Center for Social Sciences and the Bioinformatics cluster. 

\section{Third-party contributors}

\subsection{Replicators}
\label{app:3rdparty}

We are grateful to the  third-party replicators who assisted us with verifications when we were unable to access data or, in some cases, computing resources. 
%We do not name individuals when doing so would reveal information not already known to the manuscript's authors, naming the institution instead. Names are listed in no particular order.
%
Research assistants at Princeton, University of Calgary, and the Federal Reserve Board helped us. We thank Philipp vom Berge at the \ac{IAB} and  Paulo Guimarães at \ac{BPLIM} who contributed their own and their staff's time to  run code  on confidential data and provide us with detailed knowledge about the data being used. We in particular want to again thank Olivier Akmansoy, Christophe Hurlin (Université d'Orléans), and Christophe Pérignon (HEC Paris), all of \href{https://cascad.tech}{cascad}, a certification agency for scientific code and data, who have been generous of their time and resources, and have provided us with multiple reports during this time.  

We do not name the authors with whom we signed non-disclosure agreements, or who otherwise provided us with access to data that could not be published. We are grateful for their flexibility and patience.

\subsection{Computing resources}
\label{app:3rdparty-computing}

We are grateful to CodeOcean, NBER, WholeTale, and Harvard Business School, who all provided us with access to computing resources at no cost, and technical assistance when necessary. We use free academic resources on Github and Bitbucket. WholeTale is free to use for any academic user.


\section{Disclosures}
\label{sec:disclosure}

We received a generous compute and storage quota from \href{https://codeocean.com/}{CodeOcean}, a free license to use Stata 17 for one year in cloud applications from \href{https://stata.com/}{Stata}, and a subaward on NSF grant \href{https://nsf.gov/awardsearch/showAward?AWD_ID=1541450&HistoricalAwards=false}{1541450} ``CC*DNI DIBBS: Merging Science and Cyberinfrastructure Pathways: The Whole Tale'' from the University of Illinois to evaluate the WholeTale platform for the purpose of reproducibility verification. None of the sponsors have reviewed this preliminary assessment, or have had influence on any of the conclusions of this document. CodeOcean currently offers academic users a certain number of monthly free compute hours. WholeTale is free to use.


\section{Data and Code Availability Statement}
\label{sec:dcas}

All publicly available data and code used to generate figures and tables in this article are available \citep{report2022data,E117876V4}. Some detailed data from the editorial system, used for Table~\ref{tab:pre:rounds}, are considered confidential and cannot be made available in a way that preserves the privacy of the editorial process at this time.


\begin{flushright}
{\sc Lars Vilhuber}, \textit{Data Editor}
\end{flushright}


\FloatBarrier
% Remove or comment out the next two lines if you are not using bibtex.
%
% NOTE: Do not modify the AEADataEditor.bib manually!
%
\bibliographystyle{aea-mod}
\bibliography{paper,references,refs-zotero}

% The appendix command is issued once, prior to all appendices, if any.
\appendix

\input{appendix.tex}


\end{document}

